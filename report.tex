\documentclass{article}

\usepackage{graphicx} % Required for inserting images

\usepackage{amsmath} % AMS mathematical facilities for LATEX
\usepackage{amsfonts} % TEX fonts from the American Mathematical Society
\usepackage{bbold} % A geometric sans serif blackboard bold font, for use in mathematics;

\usepackage{float} % Improved interface for floating objects

\usepackage{listings} % The package enables the user to typeset programs (programming code) within LATEX
\lstset{language=Python}
\lstset
{ %Formatting for code in Python
    basicstyle=\footnotesize,
    numbers=left,
    stepnumber=1,
    showstringspaces=false,
    tabsize=1,
    breaklines=true,
    breakatwhitespace=false,
}

\setlength{\parindent}{0pt}
\usepackage{geometry} % Flexible and complete interface to document dimensions
\geometry{hmargin=2.5cm,vmargin=2.5cm}

\title{EDIN01 Cryptography \\ Project 1}
\author{Maxime Pakula, Sofia Boselli Graf}

\begin{document}

\maketitle

\tableofcontents

\newpage

\section{Introduction}

\section{Trial division}

\subsection{Exercise 1}

Assume that we have a large number $N$ containing 25 digits that is a product of two 12 digits prime numbers $p$ and $q$ :
$$N = pq$$

We want to use trial division to find the factorization of $N$. Meaning that we will test for $n \in \{1,\dotsc,\lfloor\sqrt{N}\rfloor\}$ if the following expression is null :
$$N \text{ mod } n$$

Without any knowledge of the values of prime numbers we can compute a lower bound for the number of tests. As the prime factors have 12 digits they exceed $10^{11}$ and therefore we get :
$$b_l = 10^{11}$$

We can also compute an upper bound. As one of the prime factor has to be smaller than $\lfloor\sqrt{N}\rfloor$ and $N$ is 25 digits long we get :
$$b_u = \lfloor\sqrt{10^{25}-1}\rfloor$$

Assuming that the the number of tests performed over time is $\omega = 10^7 \text{ test/s}$ we get the following interval for the factorization time :

\begin{table}[H]
    \centering
    \begin{tabular}{c|c|c}
         & number of tests & factorization time \\
        lower bound & $1,00*10^{11}$ & $1,00*10^{4} \text{ s } \approx 2,78 \text{ hours}$ \\
        upper bound & $ 3,16*10^{12}$ & $3,16*10^{5} \text{ s } \approx 3,66 \text{ days}$
    \end{tabular}
    \caption{Interval for factorization time without prime knowledge}
\end{table}

The trial division method without knowledge of values for prime numbers can take quite a long time to perform factorization. To factorize a 25 digits long number - if both prime factors are 12 digits long - can take from several hours to several days.

\subsection{Exercise 2}

Now let's assume that we have access to the knowledge of prime numbers. The new lower bound is :

$$b_l'=\frac{b_l}{ln(b_l)}=3,95*10^9$$

In the same way we can compute the new upper bound :

$$b_u'=1,25506*\frac{b_u}{ln(b_u)}=1,38*10^{11}$$

We therefore get this new following interval for the factorisation time :

\begin{table}[H]
    \centering
    \begin{tabular}{c|c|c|c}
         & number of tests & factorization time & Improvement \\
        lower bound & $3,95*10^9$ & $3,95*10^{2} \text{ s } \approx 6,58 \text{ minutes}$ & $25,3$x faster \\
        upper bound & $1,38*10^{11}$ & $1,38*10^{4} \text{ s } \approx 3,83 \text{ hours}$ & $22,9$x faster
    \end{tabular}
    \caption{Interval for factorization time with prime knowledge}
\end{table}

One of the prime numbers requires $\lceil\frac{b_u}{8*ln(2)}\rceil=6$ Bytes to be stored as the maximal value it can take is $b_u$. In order to store the slightly less than $b_u'$ prime numbers the storage required is around $827$GB. This would typically correspond to a storage price of roughly $700$ SEK, which is closer to student budget than big government grant.

\section{More efficient methods}

\subsection{Exercise 3}

\subsection{Exercise 4}

\section{The Quadratic Sieve algorithm}

\section{Appendix}

\end{document}
