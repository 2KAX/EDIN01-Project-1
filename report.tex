\documentclass{article}

\usepackage{graphicx} % Required for inserting images

\usepackage{amsmath} % AMS mathematical facilities for LATEX
\usepackage{amsfonts} % TEX fonts from the American Mathematical Society
\usepackage{bbold} % A geometric sans serif blackboard bold font, for use in mathematics;

\usepackage{float} % Improved interface for floating objects

\usepackage{listings} % The package enables the user to typeset programs (programming code) within LATEX
\lstset{language=Python}
\lstset
{ %Formatting for code in Python
    basicstyle=\footnotesize,
    numbers=left,
    stepnumber=1,
    showstringspaces=false,
    tabsize=1,
    breaklines=true,
    breakatwhitespace=false,
}

\setlength{\parindent}{0pt}
\usepackage{geometry} % Flexible and complete interface to document dimensions
\usepackage{todonotes}
\geometry{hmargin=2.5cm,vmargin=2.5cm}

\title{EDIN01 Cryptography \\ Project 1}
\author{Maxime Pakula, Sofia Boselli Graf}

\begin{document}

\maketitle

\tableofcontents

\newpage


\section{Trial division}

\subsection{Exercise 1}

Assume that we have a large number $N$ containing 25 digits that is a product of two 12 digits prime numbers $p$ and $q$ :
$$N = pq$$

We want to use trial division to find the factorization of $N$. Meaning that we will test for $n \in \{1,\dotsc,\lfloor\sqrt{N}\rfloor\}$ if the following expression is null :
$$N \text{ mod } n$$

Without any knowledge of the values of prime numbers we can compute a lower bound for the number of tests. As the prime factors have 12 digits they exceed $10^{11}$ and therefore we get :
$$b_l = 10^{11}$$

We can also compute an upper bound. As one of the prime factor has to be smaller than $\lfloor\sqrt{N}\rfloor$ and $N$ is 25 digits long we get :
$$b_u = \lfloor\sqrt{10^{25}-1}\rfloor$$

Assuming that the the number of tests performed over time is $\omega = 10^7 \text{ test/s}$ we get the following interval for the factorization time :

\begin{table}[H]
    \centering
    \begin{tabular}{c|c|c}
         & number of tests & factorization time \\
        lower bound & $1,00*10^{11}$ & $1,00*10^{4} \text{ s } \approx 2,78 \text{ hours}$ \\
        upper bound & $ 3,16*10^{12}$ & $3,16*10^{5} \text{ s } \approx 3,66 \text{ days}$
    \end{tabular}
    \caption{Interval for factorization time without prime knowledge}
\end{table}

The trial division method without knowledge of values for prime numbers can take quite a long time to perform factorization. To factorize a 25 digits long number - if both prime factors are 12 digits long - can take from several hours to several days.

\subsection{Exercise 2}

Now let's assume that we have access to the knowledge of prime numbers. The new lower bound is :

$$b_l'=\frac{b_l}{ln(b_l)}=3,95*10^9$$

In the same way we can compute the new upper bound :

$$b_u'=1,25506*\frac{b_u}{ln(b_u)}=1,38*10^{11}$$

We therefore get this new following interval for the factorization time :

\begin{table}[H]
    \centering
    \begin{tabular}{c|c|c|c}
         & number of tests & factorization time & Improvement \\
        lower bound & $3,95*10^9$ & $3,95*10^{2} \text{ s } \approx 6,58 \text{ minutes}$ & $25,3$x faster \\
        upper bound & $1,38*10^{11}$ & $1,38*10^{4} \text{ s } \approx 3,83 \text{ hours}$ & $22,9$x faster
    \end{tabular}
    \caption{Interval for factorization time with prime knowledge}
\end{table}

Let's compute a rough estimate of the storage required to store the prime numbers. Suppose that we encode numbers on $m$ bytes then we would like $m$ to be big enough to store the biggest possible prime number that we would store (which would correspond to the prime number at the position $b'_u$). This prime number is going to be smaller than $b_u$. Therefore we get the following equation.
$$2^{8m} > b_u$$
Then the amount of bytes necessary to store this number can be calculated as follows.
$$\lceil\frac{ln(b_u)}{8*ln(2)}\rceil=6$$
If every prime number needs to be stored in 6 bytes then to store $b'_u$ primes a storage of $6*b'_u = 828$ GB is required. This would typically correspond to a storage price of roughly $700$ SEK, which is closer to student budget than big government grant.

\section{More efficient methods - The Quadratic Sieve algorithm}

\subsection{Exercise 3}

In order to obtain an efficient factorization the quadratic sieve algorithm is introduced. In this exercise a version of the algorithm was coded in order to obtain the factorization of the following number:

$$N = 248163779253588348739117$$

The proposed solution is now presented. When the code is ran, 2 inputs must be provided: N, the number to factorize and F, the factor base. The first step is to calculate every element in the factor base:

\begin{verbatim}
    factorbase,factorbaseBoudary = getPrimes(cardinalOfFactorbase)
\end{verbatim}

The function \textit{\textbf{getPrimes}} returns an array of length F with every prime number from 2 until the desired length is reached. In order to do this each number (called $pr$ in the code) is checked to be a prime, this is, every number from 2 until $\sqrt{pr}$ should be non-divisible by $pr$. If the number is prime it is added to the final array.

\begin{verbatim}
    #Function to obtain the elements in F, get the first F prime numbers
    def getPrimes(F):  
        primes = []
        pr = 2  #pr: candidate prime number, starting with 2
        isPrime = True
        while len(primes) < F: #while we have less than F numbers
            for i in range(2,int(np.floor(math.sqrt(pr)))+1):  #check numbers from 2 to sqrt(pr)
                if pr % i == 0:
                    isPrime = False  #if remaining is 0 pr is not a prime number
                    break
            if isPrime:
                primes.append(pr) #adding prime to the list
            else:
                isPrime = True
            pr += 1
        return primes, pr #returns array of prime numbers and the smooth boundary B
\end{verbatim}

The next step is to construct the binary matrix which is done with the following function:

\begin{verbatim}
    binaryMatrix, parameters, r, remainders, remainderDecompositions = ...
                                         ...createBinaryMatrix(numberToFactorize,factorbase)
\end{verbatim}

The matrix is created row by row by obtaining candidates r which will hopefully meet that $r^2 mod N$ is B-smooth. The number r is chosen with the provided function with some k and j obtained from a simple function \textbf{\textit{getNextPair}} which receives a pair (k,j) and returns another one. Then, $r^2 mod N$ is calculated and the prime factors are found by using a function called \textit{\textbf{findFactors}}. This functions goes through the primes in the factor base and returns the ones for which the reminder of it and the input number is 0. The function also returns a variable which informs if the number is B smooth or not and the corresponding binary row of the matrix, if the smooth variable is true the row is added to the matrix, these steps are repeated until the matrix reaches the desired size.  

\begin{verbatim}
# This functions create a binary matrix with  len(factorbase)+2  rows.
def createBinaryMatrix(numberToFactorize,factorbase) :
    binaryMatrix = np.array([])
    parameters = []
    numbers = []
    remainders = []
    remainderDecompositions = []
    
    numberOfBinaryRowsNeeded = len(factorbase) + 4
    
    parameterPair = (0,0)

    # Until enough rows are included in the binary matrix
    while binaryMatrix.shape[0] < numberOfBinaryRowsNeeded : 
    
        # Test if the next number is smooth
        parameterPair = getNextPair(parameterPair)
        candidate = int(np.floor(math.sqrt(parameterPair[0]*numberToFactorize)) + parameterPair[1])
        remainder = candidate**2 % numberToFactorize
        factors,isRemainderSmooth,binaryRow = findFactors(remainder,factorbase)
        if not isRemainderSmooth :
            continue
    
        # Test if the binary row is already in the binary matrix
        if len(binaryMatrix) > 0 and (np.any(np.all(binaryMatrix == binaryRow, axis=1))):
            continue            # add row if not already there 
    
        if len(binaryMatrix) == 0:
            binaryMatrix = np.array([binaryRow])
        else :
            binaryMatrix = np.vstack((binaryMatrix,binaryRow))
    
        parameters.append(parameterPair)
        numbers.append(candidate)
        remainders.append(remainder)
        remainderDecompositions.append(factors)
    return binaryMatrix, parameters, numbers, remainders, remainderDecompositions
\end{verbatim}

The next step is to solve the equation x.M = 0 in mod 2:
\begin{verbatim}
    solutions = solveEquation(binaryMatrix)
\end{verbatim}

This function provides several solutions to the above equation. 
\todo[inline]{explain solveequation}

\begin{verbatim}

# This function return several different solutions to the equation M^T * x = 0 where sz(M) = (F+2,F)
def solveEquation(binaryMatrix):
    linearCombinations = np.eye(binaryMatrix.shape[0])
    indexLastFrozenRow = -1
    indexCurrentPrime = 0
    
    while indexCurrentPrime < binaryMatrix.shape[1] :
    # Find a row that is not frozen and has 1 as value for the current prime and place 
    it under the last frozen row and freeze it
    # (if no row coresponds then skip the next step and increase the index of the current prime)
        for rowIndex in range(indexLastFrozenRow+1,binaryMatrix.shape[0]) :
            if binaryMatrix[rowIndex][indexCurrentPrime] == 1 :
                break
            if binaryMatrix[rowIndex][indexCurrentPrime] != 1 :
                indexCurrentPrime += 1
                continue
    
        tmp = binaryMatrix[indexLastFrozenRow+1].copy()
        binaryMatrix[indexLastFrozenRow+1] = binaryMatrix[rowIndex].copy()
        binaryMatrix[rowIndex] = tmp.copy()
    
        tmp = linearCombinations[indexLastFrozenRow+1].copy()
        linearCombinations[indexLastFrozenRow+1] = linearCombinations[rowIndex].copy()
        linearCombinations[rowIndex] = tmp.copy()
    
        indexLastFrozenRow += 1
    
        # For all the the none frozen rows, if they have 1 as value for the current prime 
        then add the last frozen row to it
        for rowIndex in range(indexLastFrozenRow+1,binaryMatrix.shape[0]) :
            if binaryMatrix[rowIndex][indexCurrentPrime] == 1 :
                binaryMatrix[rowIndex] += binaryMatrix[indexLastFrozenRow].copy()
                binaryMatrix[rowIndex] %= 2
                linearCombinations[rowIndex] += linearCombinations[indexLastFrozenRow].copy()
    
    solutions = linearCombinations[indexLastFrozenRow+1:binaryMatrix.shape[0]]%2
    return solutions
\end{verbatim}

Finally we obtain the results and check whether the obtained results provided the factorization numbers, this is $gcd(result,N) \neq 1 \neq N$.

\todo[inline]{finish}

\begin{verbatim}
    xList, xDecompositionList, xSquaredList, ySquaredList, ySquaredDecompositionList, yList ...
                          ...= extractResults(solutions, r, remainders, remainderDecompositions)
\end{verbatim}



The results for some of the test numbers provided and the objective number can be seen in \ref{tab:results}

\begin{table}[h]
    \centering
    \begin{tabular}{|c|c|c|}
    \hline
        Number       & Factorization   & Time \\ \hline \hline
        323          &  17 . 19        &      \\ \hline
        307561       & 457 . 673       &       \\ \hline
        31741649     & 4621 . 6869     &       \\ \hline
        3205837387   & 46819 . 68473   &       \\ \hline
        392742364277 & 534571 . 734687 &       \\ \hline
        248163779253588348739117 &     &        \\
    \hline
    \end{tabular}
    \caption{Results for the test numbers as well as the time taken for each factorization}
    \label{tab:results}
\end{table}

\todo[inline]{add times and last result}




% \subsection{Exercise 4}

\section{Appendix}

\end{document}
